\hspace*{\fill} %标题前空两行

\section{游戏导出与测试}

游戏导出是指将Unity中开发完成的游戏项目导出为可执行文件或游戏包,以便在不同平台上进行发布和分发。Unity支持导出游戏到多个平台,包括Windows、Mac、Linux、iOS、Android等。

在Unity编辑器(图6-1)中,选择菜单栏中的“File”->“Build Settings”打开构建设置窗口,在构建设置窗口中,将需要导出的场景添加到“Scenes In Build”列表中,确保它们会包含在构建中,选择目标平台“Windows”,调整分辨率为“960x600”。

\begin{figure}[H]
  \centering
  \includegraphics[width=0.9\linewidth]{pic/6-1.png}
  \caption{示例图片}
\end{figure}


\begin{table}[H]
  \renewcommand\arraystretch{2} % 如果不加这一行表格行距会太小
  \centering
  \caption{示例表格2}
  \setlength{\tabcolsep}{11mm}{% 设置每列宽度
  \begin{tabular}{cccc}
    \hline
    \textbf{编号}&\textbf{名称}&\textbf{描述}&\textbf{数据类型}\\
    \hline
    1 & Product & 农产品产品ID & Int\\
    \hline
    2 & Name & 农产品产品ID & Varchar(255) \\
    \hline
  \end{tabular}}
\end{table}

游戏测试作为软件测试的一部分,具备软件测试共同的特性,测试的目的是发现软件中存在的缺陷,测试需要测试人员按照产品行为描述来实施。游戏测试作为游戏开发中质量保证的最重要的环节,在游戏设计与开发的过程中发挥着不可替代的作用。

当我们完成游戏的导出以后,打开游戏导出的文件夹,可以浏览到游戏目录(图6-2),双击“森林跳跃.exe”进行游戏启动。
\begin{figure}[H]
  \centering
  \includegraphics[width=0.9\linewidth]{pic/6-2.png}
  \caption{示例图片}
\end{figure}
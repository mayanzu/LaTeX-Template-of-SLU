\hspace*{\fill} %标题前空两行

\section{游戏设计}

\subsection{基本操作方式}
“基本操作方式”通常指的是在游戏中玩家与游戏世界进行交互的方式,这些操作方式包括移动、跳跃,使玩家可以控制游戏角色在游戏世界中进行移动与跳跃。
在本系统中,使用键盘中A键控制角色向左进行移动,键盘中D键控制角色向右进行移动,键盘中空格键控制角色进行跳跃。
在本系统中,我们设计有二次跳跃的机制,二次跳跃是指在平台游戏或类似类型的游戏中,玩家在空中进行第二次跳跃的行为,通常,玩家在跳跃后达到最高点或在空中时,可以再次按下跳跃按钮,从而使角色在空中再次执行跳跃动作。
\subsection{游戏关卡设计}
\subsubsection{障碍物}
\begin{table}[H]
  \renewcommand\arraystretch{2} % 如果不加这一行表格行距会太小
  \centering
  \caption{示例表格2}
  \setlength{\tabcolsep}{11mm}{% 设置每列宽度
  \begin{tabular}{cccc}
    \hline
    \textbf{编号}&\textbf{名称}&\textbf{描述}&\textbf{数据类型}\\
    \hline
    1 & Product & 农产品产品ID & Int\\
    \hline
    2 & Name & 农产品产品ID & Varchar(255) \\
    \hline
  \end{tabular}}
\end{table}

\begin{table}[H]
  \renewcommand\arraystretch{2} % 如果不加这一行表格行距会太小
  \centering
  \caption{示例表格2}
  \setlength{\tabcolsep}{11mm}{% 设置每列宽度
  \begin{tabular}{cccc}
    \hline
    \textbf{编号}&\textbf{名称}&\textbf{描述}&\textbf{数据类型}\\
    \hline
    1 & Product & 农产品产品ID & Int\\
    \hline
    2 & Name & 农产品产品ID & Varchar(255) \\
    \hline
  \end{tabular}}
\end{table}
在游戏开发中,障碍物通常指的是游戏世界中的物体或结构,其目的是增加游戏的挑战性和乐趣。障碍物可以以各种形式出现,本篇论文中障碍物为陷阱,陷阱是一种设计用来捕捉或伤害玩家的障碍物。这些可以是触发式的,例如地刺、陷阱门,也可以是持续性的,例如毒气云或滚动的岩石,本篇论文中仅使用地刺作为障碍物。
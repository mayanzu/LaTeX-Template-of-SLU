\hspace*{\fill} %标题前空两行

\section{游戏编程开发}
\subsection{地图开发}
\subsubsection{地图导入}
在Unity中创建一个新的场景,在窗口-包管理器中下载素材商店购买的素材,点击导入,使素材加入到目录中,将地图资源拖拽到Unity的Project视图中,新建目录名为“Grid”,在“Grid”中新建“2D”-“对象”-“瓦片地图”-矩形起名为“Ground”,“Ground”图层的本质为“tilemap”,创建一个地图层,打开“窗口”-“2D”-“平铺调色板”(图5-1),将地图素材导入其中。
\begin{figure}[H]
  \centering
  \includegraphics[width=0.5\linewidth]{pic/5-1.png}
  \caption{平铺调色板}
\end{figure}

\subsection{地图开发}
\subsubsection{地图导入}
在Unity中创建一个新的场景,在窗口-包管理器中下载素材商店购买的素材,点击导入,使素材加入到目录中,将地图资源拖拽到Unity的Project视图中,新建目录名为“Grid”,在“Grid”中新建“2D”-“对象”-“瓦片地图”-矩形起名为“Ground”,“Ground”图层的本质为“tilemap”,创建一个地图层,打开“窗口”-“2D”-“平铺调色板”(图5-1),将地图素材导入其中。
\begin{figure}[H]
  \centering
  \includegraphics[width=0.5\linewidth]{pic/5-1.png}
  \caption{平铺调色板}
\end{figure}

\subsection{地图开发}
\subsubsection{地图导入}
在Unity中创建一个新的场景,在窗口-包管理器中下载素材商店购买的素材,点击导入,使素材加入到目录中,将地图资源拖拽到Unity的Project视图中,新建目录名为“Grid”,在“Grid”中新建“2D”-“对象”-“瓦片地图”-矩形起名为“Ground”,“Ground”图层的本质为“tilemap”,创建一个地图层,打开“窗口”-“2D”-“平铺调色板”(图5-1),将地图素材导入其中。
\begin{figure}[H]
  \centering
  \includegraphics[width=0.5\linewidth]{pic/5-1.png}
  \caption{平铺调色板}
\end{figure}

\subsection{地图开发}
\subsubsection{地图导入}
在Unity中创建一个新的场景,在窗口-包管理器中下载素材商店购买的素材,点击导入,使素材加入到目录中,将地图资源拖拽到Unity的Project视图中,新建目录名为“Grid”,在“Grid”中新建“2D”-“对象”-“瓦片地图”-矩形起名为“Ground”,“Ground”图层的本质为“tilemap”,创建一个地图层,打开“窗口”-“2D”-“平铺调色板”(图5-1),将地图素材导入其中。
\begin{figure}[H]
  \centering
  \includegraphics[width=0.5\linewidth]{pic/5-1.png}
  \caption{平铺调色板}
\end{figure}
% ================= 文档用途说明 =================
%{
%   用途:本模板为安徽三联学院本科毕业论文通用模板,适用于各专业本科毕业论文撰写
%   版本:v1.3.1-simple (2025-12-30)
%   更新内容:
%     1. 恢复简单的参考文献管理方式
%     2. 简化编译流程
%     3. 完善main.tex注释,移除其他文件注释
%   编译说明:
%     1. 使用XeLaTeX编译
%     2. 编译命令:xelatex main.tex
%     3. 无需多次编译,一次编译即可完成
%   文件结构:
%     1. 主文件:main.tex(控制整个文档结构和样式)
%     2. 封面:./cover/cover_cyber.tex(论文封面)
%     3. 摘要:./abstract/abstract.tex(中英文摘要)
%     4. 正文章节:./body/目录下的.tex文件(按章节划分)
%     5. 致谢:./acknowledge/acknowledge.tex(致谢内容)
%     6. 图片资源:./pic/目录(存放论文中使用的图片)
%   使用注意事项:
%     1. 章节标题统一使用\section、\subsection等命令
%     2. 图片插入使用\includegraphics命令,图片存放于./pic目录
%     3. 表格使用\begin{table}环境,建议使用tabularx或array宏包增强功能
%     4. 参考文献引用使用\upcite{数字}命令,数字对应参考文献条目编号
%     5. 代码块使用\begin{lstlisting}环境,支持语法高亮
%     6. 数学公式使用amsmath宏包,支持复杂数学表达式
%     7. 注释符号%需要使用转义符\%表示
%     8. 确保所有文件编码为UTF-8
%   样式说明:
%     1. 页面设置:A4纸,上下左右边距2.5cm
%     2. 字体设置:正文宋体,标题黑体,英文Times New Roman
%     3. 行间距:1.5倍行距
%     4. 页码格式:目录罗马数字,正文阿拉伯数字
%     5. 页眉页脚:正文页眉显示校名,页脚居中显示页码
%}
% =================================================


\documentclass[10pt,a4paper]{ctexart}

%-------------------------- 基础与中文支持包 --------------------------%
\usepackage{iftex}                 % 条件编译支持
\usepackage{xeCJK}                 % XeLaTeX中文环境核心支持
\usepackage{fontspec}              % 字体设置宏包

%-------------------------- 常用功能宏包 --------------------------%
\usepackage{gbt7714}               % 国标GB/T 7714参考文献样式
\usepackage{amsmath,amsthm,amssymb,amsfonts,mathrsfs,mathtools} % AMS数学公式宏包集合
\usepackage{graphicx,caption,float} % 图片插入、图表标题、浮动体控制
\usepackage{geometry,fancyhdr}     % 页面布局、自定义页眉页脚
\usepackage{appendix}              % 附录支持
\usepackage{enumerate}             % 增强的列表环境(可考虑 enumitem 宏包以获得更多自定义功能)
% \usepackage{tablefootnote}         % 表格内脚注支持(若论文中无此需求,可移除)
\usepackage{tocloft}               % 目录样式深度定制
\usepackage{setspace}              % 行间距调整(如 \linespread, \setstretch)
% \usepackage{mdframed}              % 创建带边框的文本环境(功能较 framed 更强大)
% \usepackage{framed}                % 基础边框环境,若 mdframed 已满足需求则此项可注释
\usepackage{array}                 % 增强LaTeX表格和数组功能
% \usepackage{xlop}                  % 排版竖式数学运算(若论文中无此需求,可移除)
\usepackage{tabularx}              % 创建可以自动调整列宽的表格
\usepackage{xcolor}                % 颜色支持(已包含 color 宏包功能)
\usepackage[boxed,ruled,lined,linesnumbered]{algorithm2e} % algorithm2e算法排版宏包
% \usepackage{algorithmic}           % 旧版算法排版宏包,通常与 algorithm2e 任选其一
\usepackage{listings}              % 代码块排版
\usepackage{lstautogobble}         % 自动去除代码块前导空白
\usepackage{makecell}              % 表格单元格内文本换行和自定义格式
% \usepackage{underscore}            % 正确处理文本中的下划线字符

%-------------------------- 引用与链接宏包 --------------------------%
\usepackage[colorlinks,linkcolor=black,anchorcolor=black,citecolor=black]{hyperref} % 创建文档内外的超链接,建议放在许多其他包之后加载
\usepackage[symbol]{footmisc}       % 自定义脚注标记样式(例如使用符号)
\usepackage{cleveref}              % 智能交叉引用(如 \cref, \Cref)

%-------------------------- 全局字体与引用样式设置 --------------------------%
\setmainfont{Times New Roman}      % 设置文档主体英文字体为 Times New Roman
\setcitestyle{numbers,square}      % 设置参考文献引用样式为带方括号的数字(gbt7714 可能已处理)
\newcommand{\upcite}[1]{$^{[#1]}$} % 自定义上标引用命令 \upcite{数字},直接使用数字作为参考文献编号

%----------------------------- 目录格式调整 (tocloft) --------------------------------%
\setlength\cftbeforesecskip{4pt}        % 章标题在目录中的上方间距
\setlength\cftbeforesubsecskip{4pt}     % 节标题在目录中的上方间距
\setlength\cftbeforesubsubsecskip{4pt}  % 小节标题在目录中的上方间距

\renewcommand{\cftsecleader}{\cftdotfill{0.6}}       % 章条目引导点样式
\renewcommand{\cftsubsecleader}{\cftdotfill{0.6}}    % 节条目引导点样式
\renewcommand{\cftsubsubsecleader}{\cftdotfill{0.6}} % 小节条目引导点样式

\renewcommand{\cftsecfont}{\heiti \zihao{-4}}        % 章条目字体:黑体,小四
\renewcommand{\cftsecpagefont}{\songti}              % 章条目页码字体:宋体

\renewcommand{\cftsecfillnum}[1]{%                  % 章条目页码格式化
  {\cftsecleader}\nobreak
  \makebox{\cftsecpagefont #1}\cftsecafterpnum\par
}

\let\origcftsecpagefont\cftsecpagefont             % 保存原始页码字体设置
\let\origcftsecafterpnum\cftsecafterpnum           % 保存原始页码后缀设置
\renewcommand{\cftsecpagefont}{\kaishu\origcftsecpagefont} % 修改章条目页码字体为楷书(在原有宋体基础上)
\renewcommand{\cftsecafterpnum}{\origcftsecafterpnum}

\renewcommand{\cftsubsecfont}{\songti \zihao{-4}}     % 节条目字体:宋体,小四
\renewcommand{\cftsubsecfillnum}[1]{%                % 节条目页码格式化
  {\cftsecleader}\nobreak
  \makebox{\cftsecpagefont #1} \cftsecafterpnum\par
}
\renewcommand{\cftsubsubsecfont}{\songti \zihao{-4}}  % 小节条目字体:宋体,小四
\renewcommand{\cftsubsubsecfillnum}[1]{%             % 小节条目页码格式化
  {\cftsecleader}\nobreak
  \makebox{\cftsecpagefont #1}\cftsecafterpnum\par
}
%----------------------------- 目录格式调整完成 --------------------------------%

%---------------------------- 代码块颜色定义 (listings) ----------------------------%
% 这些颜色主要供 listings 包配置使用,或用于其他自定义代码片段高亮
\definecolor{codegreen}{rgb}{0,0.6,0}       % 绿色(可用于自定义,当前未直接使用)
\definecolor{codegray}{rgb}{0.5,0.5,0.5}     % 灰色 (用于 listings 行号)
\definecolor{codepurple}{rgb}{0.58,0,0.82}   % 紫色 (用于 listings 字符串)
\definecolor{backcolour}{rgb}{0.95,0.95,0.92} % 浅背景色(可用于自定义代码块背景,当前未使用)

%---------------------------- 代码块格式设置 (listings) ----------------------------%
\counterwithin{figure}{section} % 图表编号与章节关联
\lstset{
  commentstyle= \color{red!50!green!50!blue!50},  % 注释样式:混合灰色
  keywordstyle= \color{blue!70},                  % 关键字样式:蓝色
  numberstyle=\tiny\color{codegray},              % 行号样式:小号字体,灰色
  stringstyle=\color{codepurple},                 % 字符串样式:紫色
  basicstyle= \linespread{1.2}\ttfamily\footnotesize, % 基本样式:1.2倍行距, 等宽字体, 小五号
  breakatwhitespace=false,                        % 不只在空白处换行
  breaklines=true,                                % 自动长行打断
  captionpos=b,                                   % 代码块标题位置(b: 底部)
  keepspaces=true,                                % 保留代码中的空格
  numbers=left,                                   % 行号位置:左侧
  numbersep=5pt,                                  % 行号与代码的间距
  showspaces=false,                               % 不显示空格符号
  showstringspaces=false,                         % 不显示字符串中的空格符号
  showtabs=false,                                 % 不显示制表符符号
  tabsize=2,                                      % 一个制表符代表的空格数
  frame=single,                                   % 边框样式:单线边框
  autogobble=true                                 % 自动移除代码块每行共有的前导空白
}
%---------------------------- 代码块格式设置完成 ----------------------------%

%-------------------------------- 全局行间距设置 -----------------------------------%
\linespread{1.5} % 设置文档主要部分的行间距为1.5倍
%-------------------------------- 全局行间距设置完成 -----------------------------------%

%--------------------------------- 页面边距设置 (geometry) -------------------------------------%
\geometry{left=2.5cm,right=2.5cm,top=2.5cm,bottom=2.5cm} % 设置A4纸的上下左右页边距为2.5cm
%--------------------------------- 页面边距设置完成 ------------------------------------%

%-------------------------- 页眉页脚样式定义 (fancyhdr) -------------------------------%
\setlength{\headheight}{15pt} % 设置页眉高度,以容纳页眉内容

\fancypagestyle{plain}{ % 定义 "plain" 页面样式(用于章节首页、空白页)
  \fancyhf{}                   % 清空所有页眉页脚域
  \fancyfoot[C]{\zihao{5}\thepage} % 页脚居中:五号字体显示页码
  \renewcommand\headrulewidth{0pt} % 无页眉分割线
  \renewcommand\footrulewidth{0pt} % 无页脚分割线
}
\fancypagestyle{body}{ % 定义 "body" 页面样式(用于正文和目录页)
  \fancyhf{}                   % 清空所有页眉页脚域
  \fancyfoot[C]{\zihao{5}\thepage} % 页脚居中:五号字体显示页码
  \fancyhead{\centering\zihao{5}\textcolor{gray}{安徽三联学院毕业设计(论文)}} % 页眉居中:五号灰色校名论文名
  \renewcommand\headrulewidth{0.6pt} % 页眉分割线宽度0.6pt
  \renewcommand\footrulewidth{0pt} % 无页脚分割线
}
%-------------------------- 页眉页脚样式定义完成 -------------------------------%

%-------------------------- 章节标题格式定义 (ctexset) -------------------------------%
\ctexset{
  section={ % 一级标题 (章)
    format+=\heiti\zihao{3},             % 格式:黑体三号
    beforeskip=1\baselineskip,           % 标题前间距
    afterskip=1\baselineskip,            % 标题后间距
  },
  subsection={ % 二级标题 (节)
    format=\heiti\zihao{4},              % 格式:黑体四号
    beforeskip=0\baselineskip,           % 标题前间距
    afterskip=0\baselineskip,            % 标题后间距
  },
  subsubsection={ % 三级标题 (小节)
    format=\hspace*{2\ccwd}\heiti\zihao{-4}, % 格式:缩进2汉字宽度,黑体小四
    beforeskip=0\baselineskip,           % 标题前间距
    afterskip=0\baselineskip,            % 标题后间距
  },
  autoindent=true, % 标题后的第一个段落自动首行缩进
  contentsname={\hfill{\heiti\zihao{-2}目\qquad 录}\hfill}, % “目录”标题格式:黑体二号,居中
  bibname={\hfill\heiti\zihao{-2}{参考文献}\hfill},        % “参考文献”标题格式:黑体二号,居中
}
%-------------------------- 章节标题格式定义完成 ------------------------------%

%------------------------- 图表标题与编号格式定义 (caption) --------------------------%
\renewcommand{\thefigure}{\arabic{section}-\arabic{figure}} % 图编号格式:章节号-图序号
\numberwithin{table}{section} % 表格编号与章节关联
\renewcommand{\thetable}{\thesection-\arabic{table}} % 表格编号格式:章节号-表序号(使用\thesection更稳健)

\DeclareCaptionLabelSeparator{quad}{\quad} % 定义标题编号和标题文本之间的分隔符为一个\quad空格
\captionsetup{labelsep=quad} % 应用此分隔符

\captionsetup[figure]{ % 图标题设置
  font={bf,normalsize} % 字体:黑体,正常大小 (通常对应小四或五号)
}
\captionsetup[table]{ % 表标题设置
  font={bf,normalsize} % 字体:黑体,正常大小
}
%------------------------- 图表标题与编号格式定义完成 --------------------------%

%-------------------- LaTeX 文档正文开始 --------------------------%
\begin{document}

\thispagestyle{empty} % 封面页不使用任何页眉页脚
\input{"cover/cover_cyber"} %----------- 封面页 -----------%

% 摘要页,如果摘要内容排版出现问题(如断行不佳),可考虑在此局部使用 \sloppy 命令
\newpage
\input{"abstract/abstract"} %----------- 摘要页 -----------%
\newpage


\pagestyle{body}      % 从目录开始,应用 "body" 页面样式
\pagenumbering{Roman} % 目录部分使用大写罗马数字页码
\setcounter{page}{1}  % 将页码计数器重置为1(针对罗马数字部分)
\tableofcontents      % 生成目录
\clearpage            % 目录结束后换页

\pagenumbering{arabic} % 正文部分使用阿拉伯数字页码
\setcounter{page}{1}   % 将页码计数器重置为1(针对阿拉伯数字部分)
% \pagestyle{body} % 通常在 \pagenumbering 之后会自动应用或已在目录前通过 \pagestyle{body} 设置,此行可省略

% 正文各章节,如果特定章节内容排版困难,可考虑在该章节 .tex 文件内部局部使用 \sloppy
\hspace*{\fill} %标题前空两行

\section{游戏的简介}

\subsection{背景介绍}

游戏开发是一个富有创新和创造性的领域\upcite{1},游戏是一种基于规则的活动\upcite{2},通常涉及竞争、策略、目标和娱乐成分,这些规则为游戏提供结构和目的,游戏可以在不同的载体上发生,如电子游戏、桌面游戏、体育比赛、棋类等,不仅是为了娱乐和休闲,也有助于培养智力、增进交流和技能训练。

游戏作为一种娱乐形式,早已成为人们放松、减压的方式之一,游戏可以通过设计富有情感的故事情节和角色,为人们提供发泄情绪、缓解压力的出口,同时也可以给予积极向上的精神支持和疏导\upcite{3}。

\subsection{2D横版冒险游戏简介}
2D横版冒险游戏是一种经典的游戏类型,常常让玩家控制一个角色从左到右或者从右到左地移动,同时探索环境、战斗敌人、收集物品,并解决谜题以完成游戏目标,玩家通过操作主角跳跃、攻击或者其他动作,穿越各种环境场景,并解决各种谜题和障碍。

横版游戏作为最古老的动作游戏类型之一,从诞生到现在,一直都在持续不断出现各种作品,横版游戏的数量在早年的街机平台上体现得最为明显,以街机游戏而论,玩家幼年时在街机厅接触最多的除了横版格斗游戏就是横版动作游戏,掌机流行的时候,一款名为《超级马里奥》的游戏问世,横版冒险游戏逐渐走入大众的视野。

\subsection{研究目的及意义}

本论文旨在利用Unity开发引擎,设计并实现一款2D横版冒险游戏,展示Unity在2D游戏开发方面的优势和特点,完成游戏策划文档,明确游戏的类型,风格,玩法;完成游戏编程部分的编写,包括场景管理,角色控制等;完成游戏测试。

\subsection{Unity引擎简介}

Unity引擎是由Unity Technologies开发的一款功能强大、易用且灵活的游戏引擎,适用于各种规模和类型的游戏开发项目,是一款功能强大且广泛使用的跨平台游戏引擎,有用户友好的界面和易学的工作流程,适合新手和专业开发者使用,它最初于2005年推出,现已成为游戏开发行业中最受欢迎的引擎之一。

Unity引擎具有以下两种优点:

1.Unity提供了丰富的工具和可视化编辑器,包括场景编辑器、动画编辑器、粒子系统编辑器、UI编辑器等,这些可视化编辑器可以使游戏开发过程更加便利,同时降低了开发难度。
2.Unity引擎同时支持2D和3D游戏开发,并提供了丰富的功能和资源,包括模型导入、物理引擎、光照渲染、动画系统等,满足各种类型游戏的需求,Unity引擎本身带有素材商店,方便用户进行素材取用。

\subsection{2D横版冒险游戏简介}
2D横版冒险游戏是一种经典的游戏类型,常常让玩家控制一个角色从左到右或者从右到左地移动,同时探索环境、战斗敌人、收集物品,并解决谜题以完成游戏目标,玩家通过操作主角跳跃、攻击或者其他动作,穿越各种环境场景,并解决各种谜题和障碍。
横版游戏作为最古老的动作游戏类型之一,从诞生到现在,一直都在持续不断出现各种作品,横版游戏的数量在早年的街机平台上体现得最为明显,以街机游戏而论,玩家幼年时在街机厅接触最多的除了横版格斗游戏就是横版动作游戏,掌机流行的时候,一款名为《超级马里奥》的游戏问世,横版冒险游戏逐渐走入大众的视野。

\subsubsection{文章组织结构及章节安排}

清晨的森林被薄雾笼罩,露珠悬在叶片边缘,折射出微弱的晨光。科学家们在此架设了无线传感器网络,实时监测湿度、温度与生物活动。通过AI算法分析数据,他们发现某些植物的生长周期与气候变化存在非线性关联。这种技术不仅为生态研究提供新视角,也让“智慧林业”从概念走向实践——或许未来,每一棵树都将成为自然与数字世界的接口。

地铁通道里,一段即兴的爵士鼓表演吸引了人群。少年用塑料桶、铁罐和旧键盘组装成“乐器”,节奏时而急促如暴雨,时而舒缓如溪流。路人用手机记录并上传至短视频平台,24小时内点击量突破百万。评论区里,有人感叹艺术无需昂贵工具,也有人讨论城市公共空间如何包容这种自发创作——文化在裂缝中生长,反而更具韧性与共鸣。

实验室中,3D打印机正逐层堆叠由豌豆蛋白与藻类提取物制成的“牛排”。纹理模拟大理石油花,口感通过分子料理技术接近真实肉类。与此同时,垂直农场的LED灯下,紫色生菜因特定光谱照射富含三倍花青素。消费者调研显示,62\%的年轻人愿意为这类可持续食品支付溢价——吃,正在从生存行为演变为一场关于伦理与创意的实验。        %----------- 第一章 -----------%
\newpage\hspace*{\fill} %标题前空两行
\section{基础概念}

\subsection{Unity发展史}
Unity是一个跨平台的游戏开发引擎。它是由Unity Technologies公司开发,于2005年6月在苹果全球开发者大会上作为Mac OS X游戏引擎首次发布。

Unity现在已经支持桌面端、移动端、游戏主机、增强现实和虚拟现实平台。尤其在iOS和Android游戏开发方面,颇受开发者青睐,其对于初学者来说容易使用,并且在独立游戏开发中很受欢迎。

当然,Unity除了可以开发3D和2D游戏外,也可以创建互动解决方案。除了视频游戏,诸如电影、汽车、建筑、工程、制造等行业,甚至美国军队都采用了Unity的引擎。

Ambrosia Software于 2005 年 3 月发布了 Gooball。这是一款看起来远远超前于时代的游戏,制作成本仅为其他游戏的一小部分。这款游戏也为 Unity 团队提供了在正式发布之前彻底检查其引擎、识别错误、解决用户烦恼并改进界面的机会。

\subsection{插入表格}

表格示例如下:
\begin{table}[H]
  \renewcommand\arraystretch{2} % 如果不加这一行表格行距会太小
  \centering
  \caption{示例表格1}
  \setlength{\tabcolsep}{5.9mm}{% 设置每列宽度
  \begin{tabular}{cccccc}
    \hline
    \textbf{编号}&\textbf{名称}&\textbf{描述}&\textbf{数据类型}&\textbf{主键}&\textbf{备注}\\
    \hline
    1 & Product & \thead{农产品 \\ 产品ID} & Int & \checkmark & 自增长\\
    \hline
    2 & Name & 农产品产品ID & Varchar(255) &  & \\
    \hline
  \end{tabular}}
\end{table}

\begin{table}[H]
  \renewcommand\arraystretch{2} % 如果不加这一行表格行距会太小
  \centering
  \caption{示例表格2}
  \setlength{\tabcolsep}{11mm}{% 设置每列宽度
  \begin{tabular}{cccc}
    \hline
    \textbf{编号}&\textbf{名称}&\textbf{描述}&\textbf{数据类型}\\
    \hline
    1 & Product & 农产品产品ID & Int\\
    \hline
    2 & Name & 农产品产品ID & Varchar(255) \\
    \hline
  \end{tabular}}
\end{table}

\subsection{插入图片}
\subsubsection{示例}
相机视角(Camera View)在游戏开发中是指玩家或观察者在游戏中所看到的场景的视角或视觉角度。在Unity或其他游戏开发引擎中,可以通过设置摄像机的位置、旋转和投影方式来定义相机视角。

为了实现屏幕随着主角移动而移动,需要将相机与角色相绑定,采用最简单的绑定方法,导入“Main Camera”到角色文件上,调整相机位置,这样在角色移动的时候,相机会随着角色进行移动。

相机视角(Camera View)在游戏开发中是指玩家或观察者在游戏中所看到的场景的视角或视觉角度。在Unity或其他游戏开发引擎中,可以通过设置摄像机的位置、旋转和投影方式来定义相机视角。

为了实现屏幕随着主角移动而移动,需要将相机与角色相绑定,采用最简单的绑定方法,导入“Main Camera”到角色文件上,调整相机位置,这样在角色移动的时候,相机会随着角色进行移动。相机视角(Camera View)在游戏开发中是指玩家或观察者在游戏中所看到的场景的视角或视觉角度。在Unity或其他游戏开发引擎中,可以通过设置摄像机的位置、旋转和投影方式来定义相机视角。

为了实现屏幕随着主角移动而移动,需要将相机与角色相绑定,采用最简单的绑定方法,导入“Main Camera”到角色文件上,调整相机位置,这样在角色移动的时候,相机会随着角色进行移动。相机视角(Camera View)在游戏开发中是指玩家或观察者在游戏中所看到的场景的视角或视觉角度。在Unity或其他游戏开发引擎中,可以通过设置摄像机的位置、旋转和投影方式来定义相机视角。


\begin{figure}[H]
  \centering
  \includegraphics[width=0.8\linewidth]{pic/2-1.png}
  \caption{示例图片}
\end{figure}

\subsubsection{插入代码}
在角色身上创建一个新脚本,命名为“donghua.cs”,代码如下。

\begin{lstlisting}[language=java]
namespace ClearSky
{
    public class WizDemo1 : MonoBehaviour
    {
        Animator anim;
        // 在游戏对象激活时调用,初始化动画控制器
        void Awake()
        {
            anim = GetComponent<Animator>();
        }
        // 重置所有动画状态
        void ResetAnimation()
        {
            // 将所有布尔值重置为false
            anim.SetBool("isLookUp", false);
            anim.SetBool("isRun", false);
            anim.SetBool("isJump", false);
        }
        // 角色站立动画
        public void Idle()
        {
            ResetAnimation();
            anim.SetTrigger("idle");
        }
        // 角色受伤动画
        public void Hurt()
        {
            ResetAnimation();
            anim.SetTrigger("hurt");
        }
        // 角色死亡动画
        public void Die()
        {
            ResetAnimation();
            anim.SetTrigger("die");
        }
        // 角色奔跑动画
        public void Run()
        {
            ResetAnimation();
            anim.SetBool("isRun", true);
        }
        // 角色跳跃动画
        public void Jump()
        {
            ResetAnimation();
            anim.SetBool("isJump", true);
        }
    }
}
    \end{lstlisting}
        %----------- 第二章 -----------%
\newpage\hspace*{\fill} %标题前空两行

\section{美术资源和主题}

\subsection{游戏的风格和主题}

本篇论文游戏的面向群众为儿童,旨在提供儿童们轻松愉快的娱乐体验,让他们在游戏中放松心情,享受游戏带来的乐趣,可以通过有趣的方式促进儿童的学习和发展,包括认知能力、逻辑思维、创造力、想象力等,通过解决谜题、学习新知识、培养技能等。
考虑到游戏的面向群众为儿童,所以游戏选用动漫作为主要风格,动漫风格可以吸引儿童的游玩兴趣,同时动漫风格的2D横版冒险游戏可以给玩家带来愉快的游戏体验,通过可爱的角色、丰富多彩的场景和有趣的游戏内容,让玩家沉浸在一个充满想象力和趣味性的卡通世界中。

\begin{table}[H]
  \renewcommand\arraystretch{2} % 如果不加这一行表格行距会太小
  \centering
  \caption{示例表格2}
  \setlength{\tabcolsep}{11mm}{% 设置每列宽度
  \begin{tabular}{cccc}
    \hline
    \textbf{编号}&\textbf{名称}&\textbf{描述}&\textbf{数据类型}\\
    \hline
    1 & Product & 农产品产品ID & Int\\
    \hline
    2 & Name & 农产品产品ID & Varchar(255) \\
    \hline
  \end{tabular}}
\end{table}

\subsection{游戏素材美术筹备}
\subsubsection{角色素材}

冒险类型的游戏,自然需要一名冒险者,根据这个要求,再结合游戏的动漫风格,绘画出一名冒险者的立绘        %----------- 第三章 -----------%
\newpage\hspace*{\fill} %标题前空两行

\section{游戏设计}

\subsection{基本操作方式}
“基本操作方式”通常指的是在游戏中玩家与游戏世界进行交互的方式,这些操作方式包括移动、跳跃,使玩家可以控制游戏角色在游戏世界中进行移动与跳跃。
在本系统中,使用键盘中A键控制角色向左进行移动,键盘中D键控制角色向右进行移动,键盘中空格键控制角色进行跳跃。
在本系统中,我们设计有二次跳跃的机制,二次跳跃是指在平台游戏或类似类型的游戏中,玩家在空中进行第二次跳跃的行为,通常,玩家在跳跃后达到最高点或在空中时,可以再次按下跳跃按钮,从而使角色在空中再次执行跳跃动作。
\subsection{游戏关卡设计}
\subsubsection{障碍物}
\begin{table}[H]
  \renewcommand\arraystretch{2} % 如果不加这一行表格行距会太小
  \centering
  \caption{示例表格2}
  \setlength{\tabcolsep}{11mm}{% 设置每列宽度
  \begin{tabular}{cccc}
    \hline
    \textbf{编号}&\textbf{名称}&\textbf{描述}&\textbf{数据类型}\\
    \hline
    1 & Product & 农产品产品ID & Int\\
    \hline
    2 & Name & 农产品产品ID & Varchar(255) \\
    \hline
  \end{tabular}}
\end{table}

\begin{table}[H]
  \renewcommand\arraystretch{2} % 如果不加这一行表格行距会太小
  \centering
  \caption{示例表格2}
  \setlength{\tabcolsep}{11mm}{% 设置每列宽度
  \begin{tabular}{cccc}
    \hline
    \textbf{编号}&\textbf{名称}&\textbf{描述}&\textbf{数据类型}\\
    \hline
    1 & Product & 农产品产品ID & Int\\
    \hline
    2 & Name & 农产品产品ID & Varchar(255) \\
    \hline
  \end{tabular}}
\end{table}
在游戏开发中,障碍物通常指的是游戏世界中的物体或结构,其目的是增加游戏的挑战性和乐趣。障碍物可以以各种形式出现,本篇论文中障碍物为陷阱,陷阱是一种设计用来捕捉或伤害玩家的障碍物。这些可以是触发式的,例如地刺、陷阱门,也可以是持续性的,例如毒气云或滚动的岩石,本篇论文中仅使用地刺作为障碍物。        %----------- 第四章 -----------%
\newpage\hspace*{\fill} %标题前空两行

\section{游戏编程开发}
\subsection{地图开发}
\subsubsection{地图导入}
在Unity中创建一个新的场景,在窗口-包管理器中下载素材商店购买的素材,点击导入,使素材加入到目录中,将地图资源拖拽到Unity的Project视图中,新建目录名为“Grid”,在“Grid”中新建“2D”-“对象”-“瓦片地图”-矩形起名为“Ground”,“Ground”图层的本质为“tilemap”,创建一个地图层,打开“窗口”-“2D”-“平铺调色板”(图5-1),将地图素材导入其中。
\begin{figure}[H]
  \centering
  \includegraphics[width=0.5\linewidth]{pic/5-1.png}
  \caption{平铺调色板}
\end{figure}

\subsection{地图开发}
\subsubsection{地图导入}
在Unity中创建一个新的场景,在窗口-包管理器中下载素材商店购买的素材,点击导入,使素材加入到目录中,将地图资源拖拽到Unity的Project视图中,新建目录名为“Grid”,在“Grid”中新建“2D”-“对象”-“瓦片地图”-矩形起名为“Ground”,“Ground”图层的本质为“tilemap”,创建一个地图层,打开“窗口”-“2D”-“平铺调色板”(图5-1),将地图素材导入其中。
\begin{figure}[H]
  \centering
  \includegraphics[width=0.5\linewidth]{pic/5-1.png}
  \caption{平铺调色板}
\end{figure}

\subsection{地图开发}
\subsubsection{地图导入}
在Unity中创建一个新的场景,在窗口-包管理器中下载素材商店购买的素材,点击导入,使素材加入到目录中,将地图资源拖拽到Unity的Project视图中,新建目录名为“Grid”,在“Grid”中新建“2D”-“对象”-“瓦片地图”-矩形起名为“Ground”,“Ground”图层的本质为“tilemap”,创建一个地图层,打开“窗口”-“2D”-“平铺调色板”(图5-1),将地图素材导入其中。
\begin{figure}[H]
  \centering
  \includegraphics[width=0.5\linewidth]{pic/5-1.png}
  \caption{平铺调色板}
\end{figure}

\subsection{地图开发}
\subsubsection{地图导入}
在Unity中创建一个新的场景,在窗口-包管理器中下载素材商店购买的素材,点击导入,使素材加入到目录中,将地图资源拖拽到Unity的Project视图中,新建目录名为“Grid”,在“Grid”中新建“2D”-“对象”-“瓦片地图”-矩形起名为“Ground”,“Ground”图层的本质为“tilemap”,创建一个地图层,打开“窗口”-“2D”-“平铺调色板”(图5-1),将地图素材导入其中。
\begin{figure}[H]
  \centering
  \includegraphics[width=0.5\linewidth]{pic/5-1.png}
  \caption{平铺调色板}
\end{figure}        %----------- 第五章 -----------%
\newpage\hspace*{\fill} %标题前空两行

\section{游戏导出与测试}

游戏导出是指将Unity中开发完成的游戏项目导出为可执行文件或游戏包,以便在不同平台上进行发布和分发。Unity支持导出游戏到多个平台,包括Windows、Mac、Linux、iOS、Android等。

在Unity编辑器(图6-1)中,选择菜单栏中的“File”->“Build Settings”打开构建设置窗口,在构建设置窗口中,将需要导出的场景添加到“Scenes In Build”列表中,确保它们会包含在构建中,选择目标平台“Windows”,调整分辨率为“960x600”。

\begin{figure}[H]
  \centering
  \includegraphics[width=0.9\linewidth]{pic/6-1.png}
  \caption{示例图片}
\end{figure}


\begin{table}[H]
  \renewcommand\arraystretch{2} % 如果不加这一行表格行距会太小
  \centering
  \caption{示例表格2}
  \setlength{\tabcolsep}{11mm}{% 设置每列宽度
  \begin{tabular}{cccc}
    \hline
    \textbf{编号}&\textbf{名称}&\textbf{描述}&\textbf{数据类型}\\
    \hline
    1 & Product & 农产品产品ID & Int\\
    \hline
    2 & Name & 农产品产品ID & Varchar(255) \\
    \hline
  \end{tabular}}
\end{table}

游戏测试作为软件测试的一部分,具备软件测试共同的特性,测试的目的是发现软件中存在的缺陷,测试需要测试人员按照产品行为描述来实施。游戏测试作为游戏开发中质量保证的最重要的环节,在游戏设计与开发的过程中发挥着不可替代的作用。

当我们完成游戏的导出以后,打开游戏导出的文件夹,可以浏览到游戏目录(图6-2),双击“森林跳跃.exe”进行游戏启动。
\begin{figure}[H]
  \centering
  \includegraphics[width=0.9\linewidth]{pic/6-2.png}
  \caption{示例图片}
\end{figure}        %----------- 第六章 -----------%
\clearpage

\addcontentsline{toc}{section}{参考文献} % 手动将“参考文献”条目添加到目录中
\begin{thebibliography}{99} % 参考文献环境,99表示参考文献条目数目的最大位数(例如,[1] 到 [99])
\heiti % 设置参考文献条目整体字体为黑体(这会影响整个环境,可能需要更细致的控制,如仅作者或标题)
\large % 设置参考文献条目整体字体大小(同上,可能需要更细致的控制)
\setlength{\itemsep}{1.5pt} % 调整各参考文献条目之间的垂直间距
  \bibitem[1]{ref1}苏金树, 张博锋, 徐昕. 基于机器学习的文本分类技术研究进展[J]. 软件学报, 2006, 17(009): 1848-1859.
  \bibitem[2]{ref2}裴洪, 胡昌华, 司小胜, 等. 基于机器学习的设备剩余寿命预测方法综述[J]. 机械工程学报, 2019, 55(8): 1-13.
  \bibitem[3]{ref3}付宇佳, 刘晓煌, 孙兴丽, 等. 近 30 年西北内陆荒漠资源大区土地利用驱动下生态系统碳储量时空变化[J]. 地质通报, 2024, 43(23): 451-462.
  \bibitem[4]{ref4}马驰, 张国群, 孙俊格, 等. 基于深度强化学习的综合电子系统重构方法[J]. 空天防御, 2024, 7(1): 63-70.
\end{thebibliography}
\normalfont % 恢复文档默认字体设置(如果在参考文献中更改了字体)

% 致谢部分
\newpage
\input{"acknowledge/acknowledge"} %----------- 致谢 -----------%

\end{document}
%-------------------- LaTeX 文档正文结束 --------------------------%
